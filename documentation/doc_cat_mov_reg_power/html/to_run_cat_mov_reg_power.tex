
% This LaTeX was auto-generated from MATLAB code.
% To make changes, update the MATLAB code and republish this document.

\documentclass{article}
\usepackage{graphicx}
\usepackage{color}

\sloppy
\definecolor{lightgray}{gray}{0.5}
\setlength{\parindent}{0pt}

\begin{document}

    
    
\section*{To run cat\_mov\_reg\_power.m}


\subsection*{Contents}

\begin{itemize}
\setlength{\itemsep}{-1ex}
   \item Credit and date
   \item Intro
   \item Repo location
   \item Dependencies:
   \item Basic usage
   \item Example 1
   \item Sort participants as provided in the paths's list and define head brain radius
   \item Advanced usage, matching the colormap for different filtering strategies
   \item Run the loop for the first time to pick the scale
   \item Re run the loop selecting your prefered color
\end{itemize}


\subsection*{Credit and date}

\begin{par}
Code developed by Oscar Miranda-Dominguez.
\end{par} \vspace{1em}


\subsection*{Intro}

\begin{par}
This function concatenate the relative contribution of power of each frequency band from multiple subjects.
\end{par} \vspace{1em}


\subsection*{Repo location}

\begin{par}
https://gitlab.com/Fair\_lab/movement\_regressors\_power\_plots
\end{par} \vspace{1em}


\subsection*{Dependencies:}

\begin{par}
Dependancies have been included in this version. Extra functions are found within this repo's folder named 'utilities'
\end{par} \vspace{1em}


\subsection*{Basic usage}

\begin{par}
The two mandatory input arguments for this function are:
\end{par} \vspace{1em}
\begin{enumerate}
\setlength{\itemsep}{-1ex}
   \item the path to the Movement Regressors files made by the pipeline, formatted as a cell of size $nx1$, where $n$ represents the number of files
   \item TR, BOLD's repetition time
\end{enumerate}


\subsection*{Example 1}

\begin{par}
To run this example, you need to have the movement regressors files used for the power analysis. We are including in this documentation Movement regressors files from 63 participants with 4 resting state scans each. Hence we have 252 (63x4=252) Movement regressors files. Furthermore, the data was processed using 3 different methods, or versions:
\end{par} \vspace{1em}
\begin{itemize}
\setlength{\itemsep}{-1ex}
   \item ver1: No filtering
   \item ver2: applying a notch filter with fixed cutting frequencies located at 0.31 and 0.43 HZ to the estimations of head movement (Movement regressors files)
   \item ver3: applying a notch filter to the estimations of head movement but selecting the filter bandwidth based on "guesing" the participant's respiration rate.
\end{itemize}
\begin{par}
The movement regressors files for the  3 versions are saved on the folders ver1, ver2, and ver3. The files paths\_v1\_native\_folder.mat, paths\_v2\_native\_folder.mat, and paths\_v3\_native\_folder.mat have the paths to those files. You might need to update those paths accordingly to the location of the files in your system
\end{par} \vspace{1em}
\begin{par}
Here is the first example:
\end{par} \vspace{1em}
\begin{par}
Adding paths \ensuremath{|} Update this accordingly to your system
\end{par} \vspace{1em}
\begin{verbatim}
path_code='P:\code\internal\utilities\OSCAR_WIP\movement_regressors_power_plots';
addpath(genpath(path_code))

% cd /mnt/max/shared/code/internal/utilities/mov_reg_power % move to the folder to save the data
f=filesep;
TR=0.8;% TR in seconds

ver=1;
filename=['paths_v' num2str(ver) '_local_folder.mat'];
load(filename)

% YOu might have to update the paths
old='/mnt/max/shared/code/internal/utilities/mov_reg_power/';
new='P:\code\internal\utilities\OSCAR_WIP\movement_regressors_power_plots\';
paths = strrep(paths,old,new);

[CLIM, ix_subject_scan,MU,SIGMA,P]=cat_mov_reg_power(paths,TR);
\end{verbatim}

        \color{lightgray} \begin{verbatim}Error using load
Unable to read file 'paths_v1_local_folder.mat'. No such file or directory.

Error in to_run_cat_mov_reg_power (line 59)
load(filename)
\end{verbatim} \color{black}
    \begin{par}
This figure shows the power spectrum of the 252 unique scans, sorted by mean frame displacement. Each subplot indicates the direction of the displacement
\end{par} \vspace{1em}


\subsection*{Sort participants as provided in the paths's list and define head brain radius}

\begin{verbatim}
brain_radius_in_mm=50; % this is the default value. Explicitly shown here to demonstrate you can provide a different one if needed (ie for babies you might want to use 45mm instead);
sort_by_mean_FD_flag=0;
tit_preffix='sorted_as_in_the_list';
[CLIM, ix_subject_scan,MU,SIGMA,P]=cat_mov_reg_power(paths,TR,...
    'brain_radius_in_mm',45,...
    'sort_by_mean_FD_flag',0,...
    'tit_preffix',tit_preffix);
\end{verbatim}


\subsection*{Advanced usage, matching the colormap for different filtering strategies}

\begin{par}
If you like to use the same colormap and scaling using from one filtering strategy on the other filtering versions, you need to run the function first using the oputput arguments of the function and then run the function again using those output arguments as input arguments
\end{par} \vspace{1em}


\subsection*{Run the loop for the first time to pick the scale}

\begin{verbatim}
CLIM=zeros(3,6,2);
IX=cell(3,1);
MU=cell(3,1);
SIGMA=cell(3,1);
P=cell(3,1);
for ver=1:3
    filename=['paths_v' num2str(ver) '_local_folder.mat'];
    load(filename)

    % YOu might have to update the paths
    old='/mnt/max/shared/code/internal/utilities/mov_reg_power/';
    new='P:\code\internal\utilities\OSCAR_WIP\movement_regressors_power_plots\';
    paths = strrep(paths,old,new);

    tit_preffix=['FNL_ver' num2str(ver) '_autoscale_'];
    [CLIM(ver,:,:), IX{ver},MU{ver},SIGMA{ver},P{ver}]=cat_mov_reg_power(paths,0.8,'tit_preffix',tit_preffix);

end
\end{verbatim}


\subsection*{Re run the loop selecting your prefered color}

\begin{verbatim}
pick=1;
same_CLIM=squeeze(CLIM(pick,:,:));




for ver=1:3
    filename=['paths_v' num2str(ver) '_local_folder.mat'];
    load(filename)

     % YOu might have to update the paths
    old='/mnt/max/shared/code/internal/utilities/mov_reg_power/';
    new='P:\code\internal\utilities\OSCAR_WIP\movement_regressors_power_plots\';
    paths = strrep(paths,old,new);

    tit_preffix=['FNL_ver' num2str(ver) '_same_scale_as_Ver1_'];
    ix=IX{pick};
    mu=MU{pick};
    sigma=SIGMA{pick};
    p=P{pick};
    cat_mov_reg_power(paths,0.8,...
        'tit_preffix',tit_preffix,...
        'clim',same_CLIM,...
        'ix_subject_scan',ix,...
        'MU',mu,...
        'SIGMA',sigma,...
        'P',p);

end
\end{verbatim}



\end{document}
    
